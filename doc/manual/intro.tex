\section{Getting started}

\subsection{Installing and running \mumax}

After unpacking the \mumax binaries, it is recommended to add the \mumax executables to your PATH by adding this line to the end of your \file{.bashrc} file:\\
\cmd{export PATH=\$PATH:/home/me/mumax2/bin} where, of course, the correct path to the location of \mumax has to be filled in.  After starting a new shell, the program can be used by running the command \cmd{mumax2 myfile.py}, where \cmd{myfile.py} is a Python input file (See \link{examples})

\subsection{Command-line flags}

The following command-line flags are provided by \mumax:
\begin{verbatim}
  -h: Print help and exit
  -v: Print version info and exit
  -test: Test CUDA and exit
  -gpu=number: Which GPUs to use. gpu=0, gpu=0:3, gpu=1,2,3, gpu=all
  -f: Force start, remove existing output directory
  -s: Be silent
  -g: Show debug output
  -w: Show warnings
  -o="file": Specify output directory
  -log="file": Specify log file
  -timeout="time": Set a maximum run time. Units s,h,d are recognized.
  -command="file": Override interpreter command
  -cpuprof="file": Write gopprof CPU profile to file
  -memprof="file": Write gopprof memory profile to file
  -sched="type": CUDA scheduling: auto|spin|yield|sync (advanced)
  -fft="number": Override the FFT implementation (advanced)
  -t: Enable timers for benchmarking (debug)
\end{verbatim}



